% Options for packages loaded elsewhere
\PassOptionsToPackage{unicode}{hyperref}
\PassOptionsToPackage{hyphens}{url}
%
\documentclass[
]{article}
\usepackage{amsmath,amssymb}
\usepackage{lmodern}
\usepackage{ifxetex,ifluatex}
\ifnum 0\ifxetex 1\fi\ifluatex 1\fi=0 % if pdftex
  \usepackage[T1]{fontenc}
  \usepackage[utf8]{inputenc}
  \usepackage{textcomp} % provide euro and other symbols
\else % if luatex or xetex
  \usepackage{unicode-math}
  \defaultfontfeatures{Scale=MatchLowercase}
  \defaultfontfeatures[\rmfamily]{Ligatures=TeX,Scale=1}
\fi
% Use upquote if available, for straight quotes in verbatim environments
\IfFileExists{upquote.sty}{\usepackage{upquote}}{}
\IfFileExists{microtype.sty}{% use microtype if available
  \usepackage[]{microtype}
  \UseMicrotypeSet[protrusion]{basicmath} % disable protrusion for tt fonts
}{}
\makeatletter
\@ifundefined{KOMAClassName}{% if non-KOMA class
  \IfFileExists{parskip.sty}{%
    \usepackage{parskip}
  }{% else
    \setlength{\parindent}{0pt}
    \setlength{\parskip}{6pt plus 2pt minus 1pt}}
}{% if KOMA class
  \KOMAoptions{parskip=half}}
\makeatother
\usepackage{xcolor}
\IfFileExists{xurl.sty}{\usepackage{xurl}}{} % add URL line breaks if available
\IfFileExists{bookmark.sty}{\usepackage{bookmark}}{\usepackage{hyperref}}
\hypersetup{
  pdftitle={Portfolio2},
  pdfauthor={Ida Elmose Brøcker},
  hidelinks,
  pdfcreator={LaTeX via pandoc}}
\urlstyle{same} % disable monospaced font for URLs
\usepackage[margin=1in]{geometry}
\usepackage{color}
\usepackage{fancyvrb}
\newcommand{\VerbBar}{|}
\newcommand{\VERB}{\Verb[commandchars=\\\{\}]}
\DefineVerbatimEnvironment{Highlighting}{Verbatim}{commandchars=\\\{\}}
% Add ',fontsize=\small' for more characters per line
\usepackage{framed}
\definecolor{shadecolor}{RGB}{248,248,248}
\newenvironment{Shaded}{\begin{snugshade}}{\end{snugshade}}
\newcommand{\AlertTok}[1]{\textcolor[rgb]{0.94,0.16,0.16}{#1}}
\newcommand{\AnnotationTok}[1]{\textcolor[rgb]{0.56,0.35,0.01}{\textbf{\textit{#1}}}}
\newcommand{\AttributeTok}[1]{\textcolor[rgb]{0.77,0.63,0.00}{#1}}
\newcommand{\BaseNTok}[1]{\textcolor[rgb]{0.00,0.00,0.81}{#1}}
\newcommand{\BuiltInTok}[1]{#1}
\newcommand{\CharTok}[1]{\textcolor[rgb]{0.31,0.60,0.02}{#1}}
\newcommand{\CommentTok}[1]{\textcolor[rgb]{0.56,0.35,0.01}{\textit{#1}}}
\newcommand{\CommentVarTok}[1]{\textcolor[rgb]{0.56,0.35,0.01}{\textbf{\textit{#1}}}}
\newcommand{\ConstantTok}[1]{\textcolor[rgb]{0.00,0.00,0.00}{#1}}
\newcommand{\ControlFlowTok}[1]{\textcolor[rgb]{0.13,0.29,0.53}{\textbf{#1}}}
\newcommand{\DataTypeTok}[1]{\textcolor[rgb]{0.13,0.29,0.53}{#1}}
\newcommand{\DecValTok}[1]{\textcolor[rgb]{0.00,0.00,0.81}{#1}}
\newcommand{\DocumentationTok}[1]{\textcolor[rgb]{0.56,0.35,0.01}{\textbf{\textit{#1}}}}
\newcommand{\ErrorTok}[1]{\textcolor[rgb]{0.64,0.00,0.00}{\textbf{#1}}}
\newcommand{\ExtensionTok}[1]{#1}
\newcommand{\FloatTok}[1]{\textcolor[rgb]{0.00,0.00,0.81}{#1}}
\newcommand{\FunctionTok}[1]{\textcolor[rgb]{0.00,0.00,0.00}{#1}}
\newcommand{\ImportTok}[1]{#1}
\newcommand{\InformationTok}[1]{\textcolor[rgb]{0.56,0.35,0.01}{\textbf{\textit{#1}}}}
\newcommand{\KeywordTok}[1]{\textcolor[rgb]{0.13,0.29,0.53}{\textbf{#1}}}
\newcommand{\NormalTok}[1]{#1}
\newcommand{\OperatorTok}[1]{\textcolor[rgb]{0.81,0.36,0.00}{\textbf{#1}}}
\newcommand{\OtherTok}[1]{\textcolor[rgb]{0.56,0.35,0.01}{#1}}
\newcommand{\PreprocessorTok}[1]{\textcolor[rgb]{0.56,0.35,0.01}{\textit{#1}}}
\newcommand{\RegionMarkerTok}[1]{#1}
\newcommand{\SpecialCharTok}[1]{\textcolor[rgb]{0.00,0.00,0.00}{#1}}
\newcommand{\SpecialStringTok}[1]{\textcolor[rgb]{0.31,0.60,0.02}{#1}}
\newcommand{\StringTok}[1]{\textcolor[rgb]{0.31,0.60,0.02}{#1}}
\newcommand{\VariableTok}[1]{\textcolor[rgb]{0.00,0.00,0.00}{#1}}
\newcommand{\VerbatimStringTok}[1]{\textcolor[rgb]{0.31,0.60,0.02}{#1}}
\newcommand{\WarningTok}[1]{\textcolor[rgb]{0.56,0.35,0.01}{\textbf{\textit{#1}}}}
\usepackage{graphicx}
\makeatletter
\def\maxwidth{\ifdim\Gin@nat@width>\linewidth\linewidth\else\Gin@nat@width\fi}
\def\maxheight{\ifdim\Gin@nat@height>\textheight\textheight\else\Gin@nat@height\fi}
\makeatother
% Scale images if necessary, so that they will not overflow the page
% margins by default, and it is still possible to overwrite the defaults
% using explicit options in \includegraphics[width, height, ...]{}
\setkeys{Gin}{width=\maxwidth,height=\maxheight,keepaspectratio}
% Set default figure placement to htbp
\makeatletter
\def\fps@figure{htbp}
\makeatother
\setlength{\emergencystretch}{3em} % prevent overfull lines
\providecommand{\tightlist}{%
  \setlength{\itemsep}{0pt}\setlength{\parskip}{0pt}}
\setcounter{secnumdepth}{-\maxdimen} % remove section numbering
\ifluatex
  \usepackage{selnolig}  % disable illegal ligatures
\fi

\title{Portfolio2}
\author{Ida Elmose Brøcker}
\date{3/31/2022}

\begin{document}
\maketitle

\begin{itemize}
\tightlist
\item
  \emph{Type:} Individual assignment
\item
  \emph{Due:} 10 April 2022, 23:59
\end{itemize}

\hypertarget{square-root-function}{%
\subsubsection{1. Square root function}\label{square-root-function}}

To compute the square root of a given positive number, I based the
\emph{R} function upon the Newton-Rhapson method.This takes the form:
\(x_1 \cong x_0 - \frac {f(x_0)}{f'(x_0)}\)

This calculates the square root by producing an approximation to the
solutions of \(f(x) = 0\) from a guess \((x_0)\) to a point \((x_1)\)
which is closer to the solution.

In this particular case we are interested in finding the square root of
a positive number which is equivalent to finding the root of
\(f(x) = x2 − \mu = 0\) as explained in Gill's book this leaves us with
the following: \(y = (x + a / x) \times 0.5\) after we have inserted the
first derivative of the function in the Newton-Rhapson equation.

The \emph{R} function is constructed similar to the example in Gill's
book where I used a While loop which is the \emph{R} equivalent to the
Do- Untill- loop. In the function I start by estimating a number x
(mu*0.5). In the while loop the Newton-Rhapson function is defined as y
and it will loop until x is equal to y. If you insert a negative number
the program will continue looping as it goes towards negative infinity.
x will never approach y and the loop will never break. If you want it to
compute square root for negative numbers you can apply the abs()
function that takes the absolute value of the negative number and
calculates the root of the positive number.

\begin{Shaded}
\begin{Highlighting}[]
\NormalTok{square\_root }\OtherTok{\textless{}{-}} \ControlFlowTok{function}\NormalTok{(mu) \{}
\NormalTok{  x }\OtherTok{\textless{}{-}}\NormalTok{ mu}\SpecialCharTok{*}\FloatTok{0.5}
  \ControlFlowTok{while}\NormalTok{ (}\ConstantTok{TRUE}\NormalTok{) \{}
\NormalTok{    y }\OtherTok{\textless{}{-}}\NormalTok{ (x }\SpecialCharTok{+}\NormalTok{ mu }\SpecialCharTok{/}\NormalTok{ x) }\SpecialCharTok{*} \FloatTok{0.5}
    \ControlFlowTok{if}\NormalTok{ (y }\SpecialCharTok{==}\NormalTok{ x) }\ControlFlowTok{break}
\NormalTok{    x }\OtherTok{\textless{}{-}}\NormalTok{ y}
\NormalTok{  \} }
  \FunctionTok{return}\NormalTok{(y)}
\NormalTok{\}}
\end{Highlighting}
\end{Shaded}

\[
\\
\]

\hypertarget{examples}{%
\paragraph{Examples}\label{examples}}

\begin{Shaded}
\begin{Highlighting}[]
\FunctionTok{square\_root}\NormalTok{(}\DecValTok{9}\NormalTok{) }
\end{Highlighting}
\end{Shaded}

\begin{verbatim}
## [1] 3
\end{verbatim}

\begin{Shaded}
\begin{Highlighting}[]
\FunctionTok{square\_root}\NormalTok{(}\DecValTok{783}\NormalTok{)}
\end{Highlighting}
\end{Shaded}

\begin{verbatim}
## [1] 27.98214
\end{verbatim}

\begin{Shaded}
\begin{Highlighting}[]
\FunctionTok{square\_root}\NormalTok{(}\DecValTok{1000}\NormalTok{) }
\end{Highlighting}
\end{Shaded}

\begin{verbatim}
## [1] 31.62278
\end{verbatim}

\begin{Shaded}
\begin{Highlighting}[]
\FunctionTok{square\_root}\NormalTok{(}\FloatTok{0.3}\NormalTok{) }
\end{Highlighting}
\end{Shaded}

\begin{verbatim}
## [1] 0.5477226
\end{verbatim}

\hypertarget{power-series-derivatives}{%
\subsection{2. Power series
derivatives}\label{power-series-derivatives}}

\hypertarget{proof-1}{%
\paragraph{Proof 1}\label{proof-1}}

Using the definition:
\(\begin{align*} \exp(x) &:= \sum_{n=0}^\infty \frac{x^n}{n!},\\[8pt] \end{align*}\)
, we want to proof that \[\newcommand{\dx}{\:\mathrm{d}x}
\newcommand{\md}{\mathrm{d}}
\newcommand{\dfdx}{\frac{\md}{\dx}}
\begin{align*}
    \dfdx \exp(x) &= \exp(x), \\[8pt] \end{align*} \]

Proof:

\(\begin{align*} \exp(x) &:= \sum_{n=0}^\infty \frac{x^n}{n!} = \sum_{n=0}^\infty \frac{1}{n!} x^{n},\\[8pt] \end{align*}\)
we use the power rule to take the derivative:
\(\ \begin{align*} \dfdx \exp(x) &= \sum_{n=1}^\infty n \frac{1}{n!} x^{n-1},\\[8pt] \end{align*}\)
\[
 \\
 \] Thus
\[\ \begin{align*} \dfdx \exp(x) &= \sum_{n=1}^\infty \frac{1}{(n-1)!} x^{n-1}\\[8pt] \end{align*}\]

We use the rule:
\(\begin{align*} \sum_{n=0}^\infty a_i + 1 &= \sum_{n=1}^\infty a_i \\[8pt] \end{align*}\)
to change the summation index:

\[\begin{align*} \dfdx \exp(x) &= \sum_{n=0}^\infty \frac{1}{n!} x^{n} = \exp(x) \\[8pt] \end{align*}\]

\[
\\
\]

\hypertarget{proof-2}{%
\paragraph{Proof 2}\label{proof-2}}

For this we want to proof that
\(\begin{align*} \dfdx \sin(x) &= \cos(x), \\[8pt] \end{align*}\) when
using the definitions :
\(\begin{align*} \sin(x) &:= \sum_{n=0}^\infty \frac{(-1)^nx^{2n+1}}{(2n+1)!} \\[8pt] \end{align*}\)
and
\(\begin{align*} \cos(x) &:= \sum_{n=0}^\infty \frac{(-1)^nx^{2n}}{(2n)!}.\end{align*}\)

To do this, we need to take the derivitive of sin(x):
\[\begin{align*}\cos(x) &= \dfdx (\sum_{n=0}^\infty \frac{(-1)^nx^{2n+1}}{(2n+1)!}) \\[8pt]
\cos(x) &= \sum_{n=0}^\infty \frac{(-1)^n x^{2n}}{(2n)!}
\end{align*}\]

\[
\\
\]

\hypertarget{proof-3}{%
\paragraph{Proof 3}\label{proof-3}}

The last proof is a trigonometric rule as well, stating that:
\[\begin{align*} \dfdx \cos(x) &= -\sin(x).
\end{align*}\] with
\(\begin{align*} \sin(x) &:= \sum_{n=0}^\infty \frac{(-1)^nx^{2n+1}}{(2n+1)!} \\[8pt] \end{align*}\)
and
\(\begin{align*} \cos(x) &:= \sum_{n=0}^\infty \frac{(-1)^nx^{2n}}{(2n)!}.\end{align*}\)

We use the power rule to take the derivitive of cos(x):
\[\begin{align*} \dfdx \cos(x) &= \sum_{n=1}^\infty (-1)^{n} 2n \frac{x^{2n-1}}{(2n)!} \\[8pt] 
&= \sum_{n=1}^\infty (-1)^{n} \frac{x^{2n-1}}{(2n-1)!} \end{align*}\]

We change the summation index:

\[\begin{align*}\dfdx \cos(x) &= \sum_{n=0}^\infty (-1)^{n+1} \frac{x^{2n+1}}{(2n+1)!} \end{align*}\]

When applying the rule: \((-1)^n =-(-1)^{n+1}\), we end up with:

\[\begin{align*}\dfdx \cos(x) &= -\sum_{n=0}^\infty (-1)^{n} \frac{x^{2n+1}}{(2n+1)!} = -\sin(x) \end{align*}\]

\end{document}
